usepackage{listings}
usepackage{color}

definecolor{dkgreen}{rgb}{0,0.6,0}
definecolor{gray}{rgb}{0.5,0.5,0.5}
definecolor{mauve}{rgb}{0.58,0,0.82}

lstset{ % Список настроек
    language=Octave,                % Язык программирования 
    numbers=left,                   % С какой стороны нумеровать
    numberstyle=tinycolor{gray},    % Стиль который будет использоваться для нумерации строк
    stepnumber=2,                   % Шаг между линиями. Если 1, то будет пронумерована каждая строка 
    numbersep=5pt,                  
    backgroundcolor=color{white},   % Цвет подложки. Вы должны добавить пакет color - usepackage{color}
    showspaces=false,               
    showstringspaces=false,         
    showtabs=false,                
    frame=single,                   % Добавить рамку
    rulecolor=color{black},        
    tabsize=2,                      % Tab - 2 пробела
    breaklines=true,                % Автоматический перенос строк
    breakatwhitespace=true,         % Переносить строки по словам
    title=lstname,                  % Показать название подгружаемого файла
    
    keywordstyle=color{blue},       % Стиль ключевых слов
    commentstyle=color{dkgreen},    % Стиль комментариев
    stringstyle=color{mauve}        % Стиль литералов
}
